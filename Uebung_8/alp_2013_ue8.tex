\documentclass[11pt,a4paper,DIV=10,]{scrartcl}
\usepackage[utf8]{inputenc}
\usepackage[ngerman]{babel}
\usepackage{amsmath}
\usepackage{amsfonts}
\usepackage{amssymb}
\usepackage{amsthm}
\usepackage{fancybox}
\usepackage{multicol}
\usepackage{graphicx}
\usepackage{float}
\usepackage{listings}
\usepackage{color}
\usepackage{colortbl}

% Define user colors using the RGB model
\definecolor{dunkelgrau}{rgb}{0.8,0.8,0.8}
\definecolor{hellgrau}{rgb}{0.95,0.95,0.95}
\definecolor{middlegray}{rgb}{0.5,0.5,0.5}
\definecolor{lightgray}{rgb}{0.8,0.8,0.8}
\definecolor{orange}{rgb}{0.8,0.3,0.3}
\definecolor{yac}{rgb}{0.6,0.6,0.1}

% Zitation und Literaturverzeichnis
\usepackage[normal,font={small,color=black}, labelfont=bf,figurename=Abb.]{caption}
\usepackage{cite}
\usepackage{url}
\bibliographystyle{unsrtnat}
\usepackage[numbers]{natbib}
%\usepackage[T1]{fontenc}

% Formatierung für das Listing
\lstset{
   basicstyle=\scriptsize\ttfamily,
   keywordstyle=\bfseries\ttfamily\color{orange},
   stringstyle=\color{green}\ttfamily,
   commentstyle=\color{middlegray}\ttfamily,
   emph={square}, 
   emphstyle=\color{blue}\texttt,
   emph={[2]root,base},
   emphstyle={[2]\color{yac}\texttt},
   showstringspaces=false,
   flexiblecolumns=false,
   tabsize=2,
   numbers=left,
   numberstyle=\tiny,
   numberblanklines=true,
   stepnumber=1,
   numbersep=10pt,
   xleftmargin=15pt
}

\begin{document}
% ==== HEADER ==== 
\subsection*{ALP4 SoSe 2013, Di. 16-18}
\section*{Lösung Übungsblatt 8}
\textbf{Christoph van Heteren-Frese (Matr.-Nr.: 4465677)} \\ \textbf{Sven Wildermann (Matr.-Nr.: 4567553)}\\
Tutor: Alexander Steen, eingereicht am \today\\
\hrule
% === HEADER END ===
\section*{Aufgabe 1}
\subsection*{a)}
Der \texttt{RWMutex} (Read-Write-Mutex) ist das Standardinstrument, mit dem mehren Prozessen das gleichzeitige Lesen einer Ressource gestattet werden kann, während der schreibende Zugriff nur einem Prozess exklusiv möglich ist. \cite[vgl.][S. 185]{Feike2010}. Will ein Prozess den Mutex für den Schreibzugriff nutzen, obwohl gerade andere Prozesse lesen, wird dieser blockiert [vgl. ebd.]. Es werden dafür vier Zugriffsfunktionen definiert: \texttt{RLock()}, \texttt{RUnlock()}, \texttt{Lock() }und \texttt{Unlock()}. 
Grundlage der Erläuterung ist folgendes kleines Beispiel:
\lstinputlisting[language=C]{rwmutex_exp.go}
\textbf{Erläuterung:} Das Reader-Writer-Mutex \texttt{rwm} ist für den gegenseitigen Ausschluss von lesenden und schreibenden Prozessen (im Sinne seiner oben genannten Funktion) zuständig. Die Funktion \texttt{get} 'verschließt' \texttt{rwm} zunächst mittels \texttt{RLock()}, so dass keine 'Schreibzugriff' mehr möglich ist.  Nachdem \texttt{balance} gelesen und ausgegeben wurde, wird das Schloss mit \texttt{RUnlock()} wieder geöffnet. 

Die Funktion \texttt{put} arbeitet nach dem gleichen Prinzip. Statt \texttt{RLock()} und \texttt{RUnlock()} kommt hier aber \texttt{Lock()} und \texttt{Unlock()} zum Einsatz, um andere Prozesse auch den lesenden Zugriff zu verweigern. 
\subsection*{b)}
Durch die oben genannte Struktur und die erläuterten Funktionen \texttt{RLock()}, \texttt{RUnlock()}, \texttt{Lock()} bzw. \texttt{Unlock()} ergibt sich unmittelbar, dass die Invariante eingehalten wird: Wenn ein Prozess als 'Schreiber' auf die Resource \texttt{balance}  mittels \texttt{put} zuzugreifen versucht, das Schloss \texttt{rwm} aber bereits durch \texttt{RLock()} oder \texttt{Lock()} verschlossen ist, wird er blockiert. Will ein 'Leser' auf die Ressource zugreifen während das Schloss verschlossen ist, gelingt ihm das nur, wenn \texttt{rwm} mittels \texttt{RLock()} verschlossen wurde, also ein 'Leser' gerade auf die Ressource zugreift. Andernfalls (ein Schreiber hat das Schloss mittels \texttt{Lock()} versperrt) wird er blockiert. 
\subsection*{c)}
Es wird das erste 'Leser-Schreiber-Problem' gelöst. \textbf{begründung:} Es können immer wieder neue 'Leser' auf die Ressource zugreifen, unabhängig davon ob bereits 'Schreiber' warten, oder nicht.
\section*{Aufgabe 2}
\subsection*{a)}
Modellierung siehe Aufgabenteil b). Begründung der Korrektheit: Für jedes Gleis an einem Bahnhof gibt es als Semaphor so genannte ``Wächter''. Diese geben inital die Gleise frei. Sobald nun ein Zug das Gleis befahren möchte, wird ein Wächter für diesen Vorgang reserviert und das Gleis von diesem Wächter im Anschluss gesperrt. Nachdem der Zug dann durchgefahren ist, wird erst das Gleis durch den Wächter wieder freigegeben und anschließend der Wächter wieder für andere Vorgänge verfügbar. Die Zugriffe auf die Variablen sind durch mutexe geschützt. 
\subsection*{b)}
\lstinputlisting{bahnhof.go}
\subsection*{c)}
Ja, bei der Implementierung kann es zu Kollisionen zweier Züge an zusammenführenden Weichen kommen, da keine Weichen sondern nur vollständig getrennte Gleise gesichert werden. Durch weitere Anpassungen (zusätzliche Wächter für Weichen) könnte dies jedoch verhindert werden. 

\section*{Aufgabe 3}
Algorithmus in Pseudocode.
\begin{lstlisting}
 // Variabeldeklarationen
 type objects struct {
     Enum groesse = {gross, mittel, klein}
     bool richtung
     time timestamp_incoming}
 mainqueue (queue)objects
 leftqueue (queue)objects
 rightqueue (queue)objects
 islocked int
 
 // Algorithmus
 func add_object(new objects){
      // 0 = von links nach rechts
      // 1 = von rechts nach links 
     if new.richtung=0{
	    leftqueue.add(new)
      }else{
	    rightqueue.add(new)
      }
      if mainqueue.isEmpty{
	     // wenn es das erste Element ist,
	     // wird die Aktualisierung angestossen
             mainqueue.add(new)
             objekte_aktualisieren()
      }else{
	    //sonst nicht
             mainqueue.add(new)
      }
  }
 func objekte_aktualisieren(){
      current objects;
      // sortiert die Elemente nach Einfuegezeit
      mainqueue.sort(timestamp)
      leftqueue.sort(timestamp)
      rightqueue.sort(timestamp)
      //ueberpruefen ob der Weg frei ist und ob ein 
      // Objekt den Weg passieren will
      if islocked == 0 && (not mainqueue.isEmpty){
	    current = mainqueue.get
	    if current.richtung==0{
		  // pruefe nach der Reihe ob es in lefftqueue 
		  // weitere Elemente gibt, die mit current
		  // auf den Weg gehen koennen. 
		  // Sende diese inkl. current los und loesche Sie aus 
		  // den queues, erhoehe jeweils islocked um die Anzahl
		  // der losgeschickten Elemente
	    }else if current.richtung==1{
		  // pruefe nach der Reihe ob es in rightqueue 
		  // weitere Elemente gibt, die mit current
		  // auf den Weg gehen koennen. 
		  // Sende diese inkl. current los und loesche Sie aus 
		  // den queues, erhoehe jeweils islocked um die Anzahl
		  // der losgeschickten Elemente
           }
	    
	    
      }
 
 }
 
 func objekt_ist_angekommen(){
      // wird vom Objekt aufgerufen
      // sobald ein Objekt angekommen ist
      islocked.lock();
      islocked--;
      islocked.unlock();
      objekte_aktualisieren();
 }
      
 
\end{lstlisting}

Dadurch dass die Queues nach der Einfügezeit der Objekte sortiert werden führt auch ein nebenläufiges einfügen zur korrekten Reihenfolge der Elemente in der Queue. Sobald entweder ein Objekt angekommen ist oder ein Objekt in die leere haupt-queue eingefügt wird, wird ``objekte-aktualisieren'' ausgeführt.
Diese Funktion überprüft erst, welches das am längsten wartende Objekt ist und dann welche Elemente aus der Queue der selben Richtung mit diesem Element möglicherweise zusammen losgeschickt werden könnten. Diese werden dann alle zusammen losgelassen (Barriere wird geöffnet). Um einen Überblick über die 
Anzahl der zur Zeit reisenden Objekte zu erhalten, wird der Counter islocked um den entsprechenden Wert erhöht. Sobald die einzelnen Elemente am Ziel angekommen sind, führen sie objekt-ist-angekommen() aus, der den Counter schrittweise wieder verringert. So wird in der Funktion objekte-aktualisieren sicher gestellt, dass keine neuen Objekte frei gelassen werden wenn noch nicht alle Objekte wieder angekommen sind. 

\section*{Aufgabe 4}
Die Implementierung in GO compiliert leider nicht erfolgreich. In Anbetracht einer pünktlichen Abgabe habe ich es nicht mehr geschafft, dies noch zu fixen. 
\lstinputlisting{matrix.go}

\bibliographystyle{agsm}
\bibliography{alp_IV}
\end{document}
