\documentclass[11pt,a4paper,DIV=10,]{scrartcl}
\usepackage[utf8]{inputenc}
\usepackage[ngerman]{babel}
\usepackage{amsmath}
\usepackage{amsfonts}
\usepackage{amssymb}
\usepackage{amsthm}
\usepackage{fancybox}
\usepackage{multicol}
\usepackage{graphicx}
\usepackage{float}
\usepackage{listings}
\usepackage{color}
\usepackage{colortbl}

% Define user colors using the RGB model
\definecolor{dunkelgrau}{rgb}{0.8,0.8,0.8}
\definecolor{hellgrau}{rgb}{0.95,0.95,0.95}
\definecolor{middlegray}{rgb}{0.5,0.5,0.5}
\definecolor{lightgray}{rgb}{0.8,0.8,0.8}
\definecolor{orange}{rgb}{0.8,0.3,0.3}
\definecolor{yac}{rgb}{0.6,0.6,0.1}

% Zitation und Literaturverzeichnis
\usepackage[normal,font={small,color=black}, labelfont=bf,figurename=Abb.]{caption}
\usepackage{cite}
\usepackage{url}
\bibliographystyle{unsrtnat}
\usepackage[numbers]{natbib}
%\usepackage[T1]{fontenc}

% Formatierung für das Listing
\lstset{
   basicstyle=\scriptsize\ttfamily,
   keywordstyle=\bfseries\ttfamily\color{orange},
   stringstyle=\color{green}\ttfamily,
   commentstyle=\color{middlegray}\ttfamily,
   emph={square}, 
   emphstyle=\color{blue}\texttt,
   emph={[2]root,base},
   emphstyle={[2]\color{yac}\texttt},
   showstringspaces=false,
   flexiblecolumns=false,
   tabsize=2,
   numbers=left,
   numberstyle=\tiny,
   numberblanklines=true,
   stepnumber=1,
   numbersep=10pt,
   xleftmargin=15pt
}

\begin{document}
% ==== HEADER ==== 
\subsection*{ALP4 SoSe 2013, Di. 16-18}
\section*{Lösung Übungsblatt 8}
\textbf{Christoph van Heteren-Frese (Matr.-Nr.: 4465677)} \\ \textbf{Sven Wildermann (Matr.-Nr.: 4567553)}\\
Tutor: Alexander Steen, eingereicht am \today\\
\hrule
% === HEADER END ===
\section*{Aufgabe 1}
\subsection*{a)}
Der \texttt{RWMutex} (Read-Write-Mutex) ist das Standardinstrument, mit dem mehren Prozessen das gleichzeitige Lesen einer Ressource gestattet werden kann, während der schreibende Zugriff nur einem Prozess exklusiv möglich ist. \cite[vgl.][S. 185]{Feike2010}. Will ein Prozess den Mutex für den Schreibzugriff nutzen, obwohl gerade andere Prozesse lesen, wird dieser blockiert [vgl. ebd.]. Es werden dafür vier Zugriffsfunktionen definiert: \texttt{RLock()}, \texttt{RUnlock()}, \texttt{Lock() }und \texttt{Unlock()}. 
Grundlage der Erläuterung ist folgendes kleines Beispiel:
\lstinputlisting[language=C]{rwmutex_exp.go}
\textbf{Erläuterung:} Das Reader-Writer-Mutex \texttt{rwm} ist für den gegenseitigen Ausschluss von lesenden und schreibenden Prozessen (im Sinne seiner oben genannten Funktion) zuständig. Die Funktion \texttt{get} 'verschließt' \texttt{rwm} zunächst mittels \texttt{RLock()}, so dass keine 'Schreibzugriff' mehr möglich ist.  Nachdem \texttt{balance} gelesen und ausgegeben wurde, wird das Schloss mit \texttt{RUnlock()} wieder geöffnet. 

Die Funktion \texttt{put} arbeitet nach dem gleichen Prinzip. Statt \texttt{RLock()} und \texttt{RUnlock()} kommt hier aber \texttt{Lock()} und \texttt{Unlock()} zum Einsatz, um andere Prozesse auch den lesenden Zugriff zu verweigern. 
\subsection*{b)}
Durch die oben genannte Struktur und die erläuterten Funktionen \texttt{RLock()}, \texttt{RUnlock()}, \texttt{Lock()} bzw. \texttt{Unlock()} ergibt sich unmittelbar, dass die Invariante eingehalten wird: Wenn ein Prozess als 'Schreiber' auf die Resource \texttt{balance}  mittels \texttt{put} zuzugreifen versucht, das Schloss \texttt{rwm} aber bereits durch \texttt{RLock()} oder \texttt{Lock()} verschlossen ist, wird er blockiert. Will ein 'Leser' auf die Ressource zugreifen während das Schloss verschlossen ist, gelingt ihm das nur, wenn \texttt{rwm} mittels \texttt{RLock()} verschlossen wurde, also ein 'Leser' gerade auf die Ressource zugreift. Andernfalls (ein Schreiber hat das Schloss mittels \texttt{Lock()} versperrt) wird er blockiert. 
\subsection*{c)}
Es wird das erste 'Leser-Schreiber-Problem' gelöst. \textbf{begründung:} Es können immer wieder neue 'Leser' auf die Ressource zugreifen, unabhängig davon ob bereits 'Schreiber' warten, oder nicht.
\section*{Aufgabe 2}
\section*{Aufgabe 3}
\section*{Aufgabe 4}
\bibliographystyle{agsm}
\bibliography{alp_IV}
\end{document}
