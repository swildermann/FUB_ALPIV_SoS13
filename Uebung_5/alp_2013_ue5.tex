\documentclass[11pt,a4paper,DIV=10,]{scrartcl}
\usepackage[utf8]{inputenc}
\usepackage[ngerman]{babel}
\usepackage{amsmath}
\usepackage{amsfonts}
\usepackage{amssymb}
\usepackage{amsthm}
\usepackage{fancybox}
\usepackage{multicol}
\usepackage{graphicx}
\usepackage{float}
\usepackage{listings}
\usepackage{color}
\usepackage{colortbl}

% Define user colors using the RGB model
\definecolor{dunkelgrau}{rgb}{0.8,0.8,0.8}
\definecolor{hellgrau}{rgb}{0.95,0.95,0.95}
\definecolor{middlegray}{rgb}{0.5,0.5,0.5}
\definecolor{lightgray}{rgb}{0.8,0.8,0.8}
\definecolor{orange}{rgb}{0.8,0.3,0.3}
\definecolor{yac}{rgb}{0.6,0.6,0.1}

% Zitation und Literaturverzeichnis
\usepackage[normal,font={small,color=black}, labelfont=bf,figurename=Abb.]{caption}
\usepackage{cite}
\usepackage{url}
\bibliographystyle{unsrtnat}
\usepackage[numbers]{natbib}
%\usepackage[T1]{fontenc}

\begin{document}
% Formatierung für das Listing
\lstset{
   basicstyle=\scriptsize\ttfamily,
   keywordstyle=\bfseries\ttfamily\color{orange},
   stringstyle=\color{green}\ttfamily,
   commentstyle=\color{middlegray}\ttfamily,
   emph={square}, 
   emphstyle=\color{blue}\texttt,
   emph={[2]root,base},
   emphstyle={[2]\color{yac}\texttt},
   showstringspaces=false,
   flexiblecolumns=false,
   tabsize=2,
   numbers=left,
   numberstyle=\tiny,
   numberblanklines=true,
   stepnumber=1,
   numbersep=10pt,
   xleftmargin=15pt
}
\subsection*{ALP4 SoSe 2013, Di. 16-18}
\section*{Lösung Übungsblatt 5}
\textbf{Christoph van Heteren-Frese (Matr.-Nr.: 4465677)} \\ \textbf{Sven Wildermann (Matr.-Nr.: 4567553)}\\
Tutor: Alexander Steen, eingereicht am \today\\
\hrule
\section*{Aufgabe 1}
\begin{enumerate}
 \item Mit Hilfe des Tiefensuche-Algorithmus kann ein Graph auf Zyklenfreiheit untersucht werden. Hierzu überprüft man mit diesem Algorithmus, ob der
 Graph G eine Rückwärtskante besitzt oder nicht. Methode: Erweitere DFS um Speicherung der Rückwärtskanten (Back-Kante / B-Kante), sobald man von ``v''
 aus auf einem markierten Knoten ``u'' trifft, hat man eine B-Kante gefunden, falls ``u'' nicht der direkte Vorgänger von ``v'' ist. 
 \item Algorithmus: 
 \begin{lstlisting}
      function ISACYCLIC(GRAPH G=(V,E)): bool {
          B:={}
          for all v in V do { marked[v]:=false; p(v):=nil}
          for all v in V do {
	      		if not marked[v] then
	      		DFS-ACYCLIC(v)
	     	 	}
          if B:={} then return false else return true
       }
 \end{lstlisting}
  Komplexität: Da jede Kante und jeder Knoten genau einmal besucht wird, beträgt die Laufzeit von Tiefensuche O(V+E), wobei V = Anzahl der Knoten und E = Anzahl der Kanten.  
\end{enumerate}
\section*{Aufgabe 2}
Die hier vorgestellten Implementierungnen ensprechen in etwa dem Algorithmus, den Horare in \cite{Hoare1974} vorgestellt hat.

Der Monitor hat vier Zugriffsfunktionen: 
\begin{tabbing}
Links \= Funktionsname\= Rechts \kill
1.\> \texttt{start\_read} \> wird durch reader aufgerufen, der lesen möchte \\
2.\> \texttt{end\_read} \>  wird durch reader aufgerufen, der lesen beendet\\
3.\> \texttt{start\_write} \> wird durch writer aufgerufen, der schreiben möchte \\
4.\> \texttt{end\_write} \> wird durch writer aufgerufen, der schreiben beendet \\
\end{tabbing}
\begin{enumerate}
\item[a)] Implemetierung in C
\item[b)] Implementierung in Go
\end{enumerate}
\section*{Aufgabe 3}
\section*{Aufgabe 4}
Das Krümelmonsterproblem kann im Prinzip durch den in \citep{Maurer.2012} gegebenen Algorithmus gelöst werden.
\bibliographystyle{agsm}
\bibliography{alp_IV}
\end{document}