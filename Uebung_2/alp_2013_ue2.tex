\documentclass[11pt,a4paper,DIV=10,]{scrartcl}
\usepackage[utf8]{inputenc}
\usepackage[ngerman]{babel}
\usepackage{amsmath}
\usepackage{amsfonts}
\usepackage{amssymb}
\usepackage{fancybox}
\usepackage{multicol}
\usepackage{graphicx}
\usepackage{color}
\usepackage{colortbl}
% Define user colors using the RGB model
\definecolor{dunkelgrau}{rgb}{0.8,0.8,0.8}
\definecolor{hellgrau}{rgb}{0.95,0.95,0.95}
\usepackage[normal,font={small,color=black}, labelfont=bf,figurename=Abb.]{caption}
\usepackage{float}
\usepackage{cite}
\usepackage{url}
\bibliographystyle{unsrtnat}
\usepackage[numbers]{natbib}
%\usepackage[T1]{fontenc}


\begin{document}
\onecolumn
\subsection*{ALP4 SoSe 2013, Di. 16-18}
\section*{Lösung Übungsblatt 2}
\textbf{Christoph van Heteren-Frese (Matr.-Nr.: 4465677), \\ Sven Wildermann (Matr.-Nr.: 4567553)}\\
Tutor: Alexander Steen, eingereicht am \today\\
\hrule

\section*{Aufgabe 1}
\subsection*{a)}
\subsection*{b)}
\subsection*{c)}
\section*{Aufgabe 2}
\subsection*{a)}
\subsection*{b)}

\section*{Aufgabe 3}
Der Algorithmus ist nicht korrekt, da es Situationen gibt, in denen mehr als ein Prozess im kritischen Abschnitt ist. 

Ein möglicher Ablauf für den erweiterten PETERSON-Algorithmus, der zu einer Situation führt, bei der zwei der drei Prozesse im kritischen Bereich sind, sieht wie folgt aus: 

\begin{itemize}
 \item Alle drei Prozesse (1,2,3) führen die Lock-Funktion bis zur dritten Zeile aus und setzen damit interested auf true und favoured auf $(p+1)\%3$ 
 \item Zuletzt hat Prozess 3 favoured auf 1 gesetzt. 
 \item Damit springt Prozess 1 in den kritischen Abschnit, da die Bedingungen für die (warte-) for-Schleife nicht mehr erfüllt sind ($1 \&\& 0 || 1  \&\& 0$)=0.
 \item Nachdem der kritische Abschnitt abgearbeitet wurde setzt der Prozess 1 sein interested auf false. 
 \item Prozess 2 prüft die Warte-Bedingung mit dem Ergebnis ($1 \&\& 0 || 0 \&\& 1$)=0,so dass Prozess 2 die Warte-Schleife verlässt und seinen kritischen Abschnitt betritt. 
 \item Während Prozess 2 nun seinen kritischen Abschnitt ausführt, prüft Prozess 3 die Warte-Bedingung:
 \item \textbf{Die Überprüfung der Warte-Bedingung ergibt ($0 \&\& 1 || 1 \&\& 0)=0$, so dass Prozess 3 nun ebenfalls den kritischen Abschnitt betritt.}
 \item Prozess 2 und Prozess 3 befinden sich nun gleichzeitig im kritischen Abschnitt. 
 \end{itemize}

Damit ist der gegenseitige Ausschluss nicht gewährleistet und die Erweiterung des Algorithmusses nicht korrekt. 

\section*{Aufgabe 4}
\subsection*{a)}
\subsection*{b)}




\end{document}