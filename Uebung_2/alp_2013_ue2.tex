\documentclass[11pt,a4paper,DIV=10,]{scrartcl}
\usepackage[utf8]{inputenc}
\usepackage[ngerman]{babel}
\usepackage{amsmath}
\usepackage{amsfonts}
\usepackage{amssymb}
\usepackage{amsthm}
\usepackage{fancybox}
\usepackage{multicol}
\usepackage{graphicx}
\usepackage{color}
\usepackage{colortbl}
% Define user colors using the RGB model
\definecolor{dunkelgrau}{rgb}{0.8,0.8,0.8}
\definecolor{hellgrau}{rgb}{0.95,0.95,0.95}
\usepackage[normal,font={small,color=black}, labelfont=bf,figurename=Abb.]{caption}
\usepackage{float}
\usepackage{cite}
\usepackage{url}
\bibliographystyle{unsrtnat}
\usepackage[numbers]{natbib}
%\usepackage[T1]{fontenc}


\begin{document}
\onecolumn
\subsection*{ALP4 SoSe 2013, Di. 16-18}
\section*{Lösung Übungsblatt 2}
\textbf{Christoph van Heteren-Frese (Matr.-Nr.: 4465677), \\ Sven Wildermann (Matr.-Nr.: 4567553)}\\
Tutor: Alexander Steen, eingereicht am \today\\
\hrule

\section*{Aufgabe 1}
\subsection*{a)}
Bei diese Implementierung ist es möglich, dass zunächst $a[0]=a[1]=false$ gilt (das Prinzip der Unteilbarkeit der Abfrage eines Zustandes und seiner Veränderung ist nicht eingehalten \cite[vgl.][S. 39]{Maurer.2012}). Im anshcließenden Schleifendurchlauf kann durch die gleiche Ablaufreihenfolge $a[0]=a[1]=true$ gelten, wodurch beide Prozesse gleichzeitig ihren kritischen Bereich betreten würden. Der gegenseitige Ausschluss ist hier also nicht gewährleistet. 
\subsection*{b)}
Bei dieser Version ist eine Ablaufreihenfolge möglich, bei der \texttt{interested[0]=true} und \texttt{interested[1]=true} gilt. Wenn beide Prozesse den kritischen Abschnit betreten möchten, wird einer dem andren immer den Vortitt lassen. Zwar ist es warscheinlich, dass sich bei einem der nächsten Schleifendurchläufe eine andere Situation einstellt, aber grundsätlich ist so eine Implementierung zu verwerfen, da die Gefahr der Verklemmung (hier: \textit{livelock}) besteht. Dijkstra schreib dazu: \begin{quote}\glqq If the two processes are about to enter their critical sections, it must be impossible to devise for them such finite speeds, that the decision which one of the two is the first to enter its critical section is postponed to eternity\grqq \\ \cite[vgl.][S. 80]{Dijkstra.1965}\end{quote}
%Verklemmung
\subsection*{c)}
Bei dieser Implementierung fällt auf, dass der Variablen \texttt{favoured} kein wert zugewiesen wird. Demnach kann -- abhängig von der Initialisierung -- nur einer der beiden Prozesse seinen kritischen Bereich betreten. Die Warteschleife des nicht (initial) favourisierten Prozesses kann so nie verlassen werden.
%Fairness
\section*{Aufgabe 2}
\subsection*{a)}
\subsubsection*{Gegenseitiger Ausschluss:}
\textbf{Behauptung:} Der Algorithmus von Dekker erfüllt den wechselseitigen Ausschluss.\\
\textbf{Beweis:} Angenommen, die Aussage ist falsch. Dann gibt es eine Ablaufreihenfolge bei der sich beide Prozesse in ihrem kritischen Abschnitt befinden. Daraus folgt, dass beide Prozesse die Austrittsbedingung  der äußeren $for$-Schleife erfüllt haben. Drei Fälle können unterschieden werden, die dazu führen könnten:
\begin{enumerate}
\item \textbf{Keiner der Beiden Prozesse läuft durch den Schleifenkörper.} Beiden Prozesse müssten dann die jeweilige Schleifenabbruchbedingung von Anfang an erfüllen ($\neg i_1$ für $P0$ und $\neg i_0$ für $P1$). Das ist aber nicht möglich. Sei $P1$ der zweite Prozess, der die $for$-Bedingung prüft. Dann hat $P0$ vorher durch die Anweisung  \texttt{interested[p]} dafür gesorgt, dass $P1$ in die Schleife eintreten muss. Aus Symetriegründen gilt dies auch für die andere Reihenfolge.
\item \textbf{Beide Prozess durchlaufen den Schleifenkörper.} Dann muss einer der beiden Prozesse in der inneren $for$-Schleife hängen bleiben, da favoured erst nach Durchlaufen des kritischen Abschnitts (durch \textit{Unlock}) geändert wird.
\item \textbf{Einer der Prozesse hat den Schleifenkörper durchlaufen während sich der andere bereits im kritischen Abschnitt befindet.} Dass geht auch nicht. Es lässt sich leicht nachprüfen, dass er den kritischen Abschnitt nicht mehr betreten kann.
\end{enumerate}
Diese Fälle zeigen, dass die Annahme falsch ist. Somit garantiert der Dekker-Algorithmus den wechselseitigen Ausschluss. $\qed$
%Zunächst lässt sich festhalten, dass jeder Prozess nur die eigene Prozessvariable ändern kann. Prozess 0 fragt $c1$ nur ab (prüft solange ob Prozess 1 interesse bekundet, bis), solange $c1 = 0$ gilt (Prozess 1 kein Interesse bekundet), betritt den kritischen Bereich aber nur, wenn $c2  = 1$ gitl. Daher können die beiden Prozesse nie gleichzeitig in ihreren kritischen Bereichen sein.
\subsection*{b)}
\subsubsection*{Verklemmungsfreiheit:}
\glqq Eine Verklemmung ist nur möglich, wenn bei Erfüllung der Vorbedingungen $i_0$ und $i_1$ für den Eintritt in die Schleifen in beiden Eintrittsprotokollen keinen von ihnen Terminiert [...]\grqq \cite[vgl.][S. 44]{Maurer.2012}. Denn: Wenn nur ein Prozess Interesse bekundet, wird an Stelle des  Schleifenrumpfes sofort der kritische Bereich betreten. Bekunden aber beide Prozesse Interesse, zieht der nicht favourisierte Prozess sein Interesse solange zurück, bis er selbst favourisiert wird und lässt dem anderen Prozess somit den Vortritt.

\subsubsection*{Fairness und Behinderungsfreiheit:}
Die Anweisung \texttt{favoured=1-p} im Austrittsprotokoll sorgt dafür, dass nach Verlassen des kritischen Bereiches eines Prozesses immer der Andere favourisiert wird und so zum Zuge kommt. Es ist aber trotzdem sichergestellt, das dieser -- sollte er als Einziger Interesse bekunden --  den kritischen Bereich erneut betreten kann, da dann der gesammte Schleifenrumpf übersprungen wird und \texttt{favoured} gar nicht zum tragen kommt. \\
\textit{Annmerkung:} Fairness impliziert Verklemmungsfreiheit. Denn wenn garantiert ist, dass der Prozess, der in den kritischen Bereich eintreten will, dies auch schließlich tut, dann ist auch garantiert, dass von den Prozessen, die in die kritische Sektion eintreten wollen, dies auch einer tut.

%\textbf{Behauptung:} Der Algorithmus von Dekker ist Verklemmungs-frei. \\
%\textbf{Beweis:} Nehmen wir an, die Aussage sei falsch. Dann betritt einer der beiden Prozesse den Initialisierungscode (Zeilen 3-7), aber nie den kritischen Abschnitt. Da sich beide Prozesse symmetrisch verhalten, nehmen wir ohne Beschränkung der Allgemeinheit an, dass P verhungert. Hierfür kann es nur zwei Gründe geben: Die while-Schleife wird unendlich oft durchlaufen, oder der Prozess P kommt nicht aus einem await-Befehl heraus (Zeile (P6)). Betrachten wir zunächst den zweiten Fall. Dann muss turn den Wert 2 haben. Wir unterscheiden die folgenden Fälle, anhand dessen, wo sich Q befindet:
%\begin{enumerate} %\item Q ist in seinem restlichen Code. Das ist unmöglich, da wantq wahr gewesen sein muss als P die while-Schleife betreten hat. Zu diesem Zeitpunkt muss Q in den Zeilen (Q2) bis (Q9) gewesen sein. Um danach wieder in (Q1) zu kommen, hätte Q den Wert turn in Zeile (Q9) auf 1 setzen müssen. Wir haben aber angenommen, dass turn den Wert 2 hat.\end{enumerate}
%• Q ist in seinem restlichen Code. Das ist unmöglich, da wantq wahr gewesen sein muss als P die while-Schleife betreten hat. Zu diesem Zeitpunkt muss Q in den Zeilen (Q2) bis (Q9) gewesen sein. Um danach wieder in (Q1) zu kommen, hätte Q den Wert turn in Zeile (Q9) auf 1 setzen müssen. Wir haben aber angenommen, dass turn den Wert 2 hat.
\section*{Aufgabe 3}
Der Algorithmus ist nicht korrekt, da es Situationen gibt, in denen mehr als ein Prozess im kritischen Abschnitt ist. Ein möglicher Ablauf für den erweiterten PETERSON-Algorithmus, der zu einer Situation führt, bei der zwei der drei Prozesse im kritischen Bereich sind, sieht wie folgt aus: 

\begin{itemize}
 \item Alle drei Prozesse (1,2,3) führen die Lock-Funktion bis zur dritten Zeile aus und setzen damit interested auf true und die globale Variable favoured auf $(p+1)\%3$ 
 \item Zuletzt hat Prozess 3 favoured auf 1 gesetzt. 
 \item Damit springt Prozess 1 in den kritischen Abschnitt, da die Bedingungen für die (warte-) for-Schleife nicht mehr erfüllt sind ($1 \&\& 0 || 1  \&\& 0$)=0.
 \item Nachdem der kritische Abschnitt abgearbeitet wurde setzt der Prozess 1 sein interested auf false. 
 \item Prozess 2 prüft die Warte-Bedingung mit dem Ergebnis ($1 \&\& 0 || 0 \&\& 1$)=0, so dass Prozess 2 die Warte-Schleife verlässt und seinen kritischen Abschnitt betritt. 
 \item Während Prozess 2 nun seinen kritischen Abschnitt ausführt, prüft Prozess 3 die Warte-Bedingung:
 \item \textbf{Die Überprüfung der Warte-Bedingung ergibt ($0 \&\& 1 || 1 \&\& 0)=0$, so dass Prozess 3 nun ebenfalls den kritischen Abschnitt betritt.}
 \item Prozess 2 und Prozess 3 befinden sich nun gleichzeitig im kritischen Abschnitt. 
 \end{itemize}

Damit ist der gegenseitige Ausschluss nicht gewährleistet und die Erweiterung des Algorithmusses nicht korrekt. 

\section*{Aufgabe 4}
\subsection*{a)}
Die Begründung der Korrektheit der ersten Version des Bäckerei-Algorithmus von Lamport erfolgt informal und abgekürzt. Der formale Beweis kann im offiziellem Paper nachgelesen werden \cite[vgl.][]{Lamport.1974}.

Jeder ``Kunde'' (oder Thread) zieht nacheinander (garantiert durch die Variable ``drawing'') eine Nummer, welche um 1 höher sein sollte als die bisher am höchsten vergebene Nummer. Der ``Verkäufer'' (Prozess) ruft dann die Kunden nacheinander auf - also immer die nächst höhere natürliche Zahl. Sobald zwei Kunden gleichzeitig eine Nummer gezogen haben und damit die gleiche Nummer erhalten haben, entscheidet der Zufall darüber wer von den Kunden mit der gleichen Nummer zuerst bedient wird. 
Weil nur n Kunden gleichzeitig eine Zahl ``ziehen'' und somit die gleiche Zahl erhalten können, ist nach (n-1)-Schritten (endlich vielen!) auch der letzte Kunde mit der selben Zahl dran. Außerdem wird von jedem Verkäufer immer nur ein Kunde bedient und die Bedienung ist nach endlicher Zeit erledigt. Somit sind gegenseitiger Ausschluss, Behinderungsfreiheit, Verklemmungsfreiheit und Fairness garantiert. Des weiteren fällt ``busy waiting'' weg, da immer ``der Nächste'' aufgerufen wird und die Kunden nicht dauernd fragen müssen, ob Sie ``der Nächste'' sind. 
\subsection*{b)}
Der Unterschied zwischen dem ersten und zweiten Algorithmus besteht darin, dass der Fall des ``gleichzeitigen Ziehens einer Nummer'' nun anders behandelt wird. Während vorher dieser Bereich durch die Variable ``drawing'' abgesichert wurde (-und alle Kunden in eine Warteschlange geschickt wurden, bis diese dann eine Nummer ziehen konnten-) , gibt es nun eine initiale Nummer (1), die jeder ``Kunde'' erhält noch bevor er seine richtige Nummer zieht. Somit kann verhindert werden, dass die Nummer eines eintrittswilligen Kunden nicht durch Unterbrechnung eines anderen auf 0 stehen bleibt und dieser somit garnicht bedient wird. 

\bibliographystyle{agsm}
\bibliography{alp_IV}

\end{document}