\documentclass[11pt,a4paper,DIV=10,]{scrartcl}
\usepackage[utf8]{inputenc}
\usepackage[ngerman]{babel}
\usepackage{amsmath}
\usepackage{amsfonts}
\usepackage{amssymb}
\usepackage{amsthm}
\usepackage{fancybox}
\usepackage{multicol}
\usepackage{graphicx}
\usepackage{float}
\usepackage{listings}
\usepackage{color}
\usepackage{colortbl}

% Define user colors using the RGB model
\definecolor{dunkelgrau}{rgb}{0.8,0.8,0.8}
\definecolor{hellgrau}{rgb}{0.95,0.95,0.95}
\definecolor{middlegray}{rgb}{0.5,0.5,0.5}
\definecolor{lightgray}{rgb}{0.8,0.8,0.8}
\definecolor{orange}{rgb}{0.8,0.3,0.3}
\definecolor{yac}{rgb}{0.6,0.6,0.1}

% Zitation und Literaturverzeichnis
\usepackage[normal,font={small,color=black}, labelfont=bf,figurename=Abb.]{caption}
\usepackage{cite}
\usepackage{url}
\bibliographystyle{unsrtnat}
\usepackage[numbers]{natbib}
%\usepackage[T1]{fontenc}

\begin{document}
\subsection*{ALP4 SoSe 2013, Di. 16-18}
\section*{Lösung Übungsblatt 4}
\textbf{Christoph van Heteren-Frese (Matr.-Nr.: 4465677), \\ Sven Wildermann (Matr.-Nr.: 4567553)}\\
Tutor: Alexander Steen, eingereicht am \today\\
\hrule
\section*{Aufgabe 1}
\section*{Aufgabe 2}
\section*{Aufgabe 3}
\section*{Aufgabe 4}
% Formatierung für das Listing
\lstset{
   basicstyle=\scriptsize\ttfamily,
   keywordstyle=\bfseries\ttfamily\color{orange},
   stringstyle=\color{green}\ttfamily,
   commentstyle=\color{middlegray}\ttfamily,
   emph={square}, 
   emphstyle=\color{blue}\texttt,
   emph={[2]root,base},
   emphstyle={[2]\color{yac}\texttt},
   showstringspaces=false,
   flexiblecolumns=false,
   tabsize=2,
   numbers=left,
   numberstyle=\tiny,
   numberblanklines=true,
   stepnumber=1,
   numbersep=10pt,
   xleftmargin=15pt
}
\subsection*{a)}
Das in \citep{Maurer.2012} erläuterte Prozessmodel unterscheidet fünf Zustände: \textit{nicht existent}, \textit{bereit}, \textit{aktiv}, \textit{blockiert} und \textit{beendet}.
Eine Semaphore kann in Java wie folgt implementiert werden:
% Semaphore-listing 
\lstinputlisting
    [caption={Implemntierung eines Semaphor in Java nach \cite{Kramann}. (Die Methoden \texttt{suspend()} und \texttt{resume()} sind veraltet und sollten in der Praxis nicht mehr verwendet werden.) }
       \label{lst:javaclass},
       captionpos=b,language=JAVA,
       ]
{listings/Semaphore.java}
\subsection*{b)}
\subsection*{c)}

\bibliographystyle{agsm}
\bibliography{alp_IV}

\end{document}